\documentclass[twocolumn]{AS4201}
\usepackage[spanish,english]{babel} % Para separar correctamente las palabras de multitud de idiomas
\usepackage[utf8]{inputenc} %Este paquete permite poner acentos directamente y eñes
\usepackage{graphicx}
\usepackage{epsfig}
\usepackage{longtable} % para tablas mas largas que un sola pagina
%%%%%%%%%%%%%%%%%%%%%%%%%%%%%%%%%%%%%%%%
\begin{document}
%
\title{Tarea $N^o$1 AS4501/CC4501}
%\subtitle{?}
\author{Crist\'an Carre\~no M.\inst{1} \and
		Diego Chavez E. \inst{1}\and
		He\'ctor Salas O.\inst{2}\\ 
\inst{}
%\fnmsep\thanks{Just to show the usage
%of the elements in the author field}
}
%\offprints{G. Wuchterl}
\institute{	
	Departamento de ciencias de la computación (DCC), Universidad de Chile
	Avenida Blanco Encalada 2120, Tercer Piso, Santiago, Chile
	%\email{c.ptolemy@hipparch.uheaven.space}
	\and
	Departamento de Astronom\'ia (DAS), Universidad de Chile,
	Camino El observatorio 1515, Las Condes, Chile\\
	%\email{wuchterl@amok.ast.univie.ac.at}
	%\thanks{The university of heaven temporarily does not5
	%accept e-mails}
}
\date{\today}
\abstract{Una de las fascetas enseñadas en este curso, es el enfoque del manejo de datos y de estándares astronómicos, y de la familiarización con los sucesos que acontecen en la captura de estos mismos datos en imágenes. Este informe alude principalmente a métodos que se efectúan en la vida real para analizar de forma correcta estas imágenes y permitir así clasificaciones y procesamientos de datos para la refinación de las mismas mediante herramientas que provee el lenguaje de programación $Python$ y algunas librerías.
}
\maketitle
%
%________________________________________________________________
\section{Introducción}
El objetivo de esta tarea es familiarizarce con el lenguaje de programación Python y con librerías usadas comunmente para procesamiento de imágenes astronómicas. El objetivo es el de generar una imagen astronómica a partir de un catálogo de objetos, al que posteriormente se le aplicarán métodos, tanto para agregar $background$, $ruido$ , una $PSF(Point Spread Function)$ para lograr una cercanía mayor a lo que sería una imagen capturada por un centro astronómico ( fuentes de errores comunes). Finalmente, a esta imagen simulada, se le aplicará un filtro, que forma parte del proceso de refinación de la imagen, y que permite en la realidad, analizar los datos disminuyendo las fuentes de errores.

\section{Metodología} 
A continuación , la lista de pasos a seguir y aplicar al proceso de generación de la imagen simulada.
\subsection{Obtencion de Cuentas desde Magnitudes}
\label{cuentas}
Lo primero que se nos pide para realizar esta tarea es la creación de la rutina "mToCounts(m,m0,F0)" que reciba el la magnitud de un objeto (m)y retorne el numero de cuentas recibidos el CCD dadas una magnitud y flujo de referencia ($m_0$,$F_0$).\\
\\
Partiendo de la ecuación:
\begin{equation}
	m-m_0=-2.5log(\frac{F}{F_0})
\end{equation}
Despejando $F_0$ se tiene:
\begin{equation}
	10^{\frac{-2(m-m_0)}{5}}=\frac{F}{F_0}
\end{equation}
Pero se sabe además que :
\begin{equation}
	F=\frac{energia}{Area*exptime}=\frac{\# cuentas}{exptime}
\end{equation}
%Luego 
%\begin{equation}
%	\frac{F}{F_0}=\frac{\frac{\# cuentas}{exptime}}{\frac{\# cuentas_0}{exptime}}=\frac{\# cuentas}{\# cuentas_0}
%\end{equation}
finalmente 
\begin{equation}
	\# Cuentas=exptime*F_0*10^{\frac{-2(m-m_0)}{5}}
\end{equation}
Notar que la función así como la hemos definido depende además del tiempo de exposición, sin embargo este lo obtendremos del header del archivo fits con el que vamos a (Key EXPTIME del header)  
\subsection{Simulación de estrellas}
A continuación se nos pidió generar la rutina "addStar(hdu,m,RA,DEC)" que reciba la magnitud y coordenadas de una estrella.\\ 	

Para esto es primero necesario convertir las coordenadas celestes $(RA,DEC)$ en coordenadas cartesianas en la imagen $(row,col)$ en la imagen; de lo visto en la clase auxiliar se sabe que se pueden obtener las coord celestes de un píxel  $(RA_p,DEC_p)$ cuyas coordenadas en la imagen son $(row_p,col_p)$, a partir de un píxel de referencia de coord (cuyas coord celestes, así como sus coord en la imagen son conocidas) m\'a s la inflacionario de como varían RA y DEC al variar row y col. \\Esto se hace mediante la siguiente ecuación: \\ 
$%\begin{equation}
 	\left[\begin{array}{c}
 	    RA_p \\DEC_p 
	\end{array}\right] 
	=
	\left[\begin{array}{cc}
		\Delta RA_{row} & \Delta RA_{col}\\ 
		\Delta DEC_{row} & \Delta DEC_{col}
	\end{array}\right] 
	\left[\begin{array}{c}
 	    row_p-row_{ref\_ pix}\\col_p-col{ref\_ pix}
	\end{array}\right] 		
%\end{equation}
$
\begin{equation}
\hspace{100pt}
+
	\left[\begin{array}{c}
 	       RA_{ref\_ pix} \\DEC{ref\_ pix}
	\end{array}\right] 
\end{equation}
La información necesaria para realizar esta transformación se encuentra en el header de la imagen en los siguientes campos:
\begin{flushleft}
	\label{table:1}      % is used to refer this table in the text
	\begin{tabular}{ll}        % centered columns (4 columns)
	KEY & valor\\
	\hline                % inserts double horizontal lines
	CRPIX1 & Columna del pixel de referencia\\
	%\hline   
	CRPIX2 & Fila del pixel de referencia  \\
	%\hline    
	CRVAL1 & RA del pixel de referencia \\
	%\hline
	CRVAL2 & DEC del pixel de referencia \\
	%\hline
	CD1\_1 & Variación de RA (en grados) por columna\\
	%\hline
	CD1\_2 & Variación de RA (en grados) por fila\\
	%\hline	  	
	CD2\_1 & Variación de DEC (en grados) por columna\\
	%\hline
	CD2\_2 & Variación de DEC (en grados) por fila\\
	%\hline		
\end{tabular}\end{flushleft}
Las que se pueden obtener haciendo uso de algunas rutinas de la librería pyfits, como se muestra a continuación\\
$\\
import pyfits 
\\
hdulist = pyfits.open('nombrearchivo.fits')\\
hdulist.close()\\
var=hdulist[0].header['KEY']\\
$\\
Remplazando esto en la ecuación anterior:
\begin{equation}\label{a}
 	\left[\begin{array}{c}
 	    RA_p \\DEC_p 
	\end{array}\right] 
	=
	\left[\begin{array}{cc}
		CD1\_2 & CD1\_1\\ 
		CD2\_2 & CD2\_1\\
	\end{array}\right] 
	\left[\begin{array}{c}
 	    row_p-CRPIX2 \\col_p-CRPIX1
	\end{array}\right] 		
\end{equation}
$
\hspace{100pt}
+
	\left[\begin{array}{c}
 	    CRVAL1 \\CRVAL2
	\end{array}\right] 
$\\
\\
Como nos interesa el problema inverso tenemos que despejar	
$\left[\begin{array}{c}
row_p \\col_p
 \end{array}\right]$ 
de la ecuación (\ref{a}), esto queda:
\begin{equation}\nonumber
 	\left[\begin{array}{c}
 	    row_p \\col_p 
	\end{array}\right] 
	=M^{-1} 	
	\left[\begin{array}{c}
 	    RA_p-CRVAL1\\DEC_p-CRVAL2
	\end{array}\right] 	
	+
	\left[\begin{array}{c}
 	    CRPIX2\\CRPIX1
	\end{array}\right] 
\end{equation}
Donde:
\begin{eqnarray}\nonumber
	M^{-1} & = &
	\left[\begin{array}{cc}
		CD1\_2 & CD1\_1\\ 
		CD2\_2 & CD2\_1\\
	\end{array}\right]^{-1} \\
	\nonumber
	& = & \frac{1}{det|M|}%{CD1\_2*CD2\_2-CD1\_1*CD2\_1}	
	\left[\begin{array}{cc}
		CD2\_1 & -CD1\_1\\ 
		-CD2\_2 & CD1\_2\\
	\end{array}\right]
\end{eqnarray}\label{estrella}
Ahora que sabemos en que coordenadas de la imagen se encuentra la estrella podemos simularla asignándole al píxel correspondiente a la ubicación de la estrella el numero de cuentas que corresponden a la estrella dada su magnitud (las estrellas se simulan como fuentes puntuales, es decir, como un solo píxel en la imagen), para esto hacemos uso de la "rutina mToCounts" definida anteriormente, usando como valor de referencia FLUX20 (el flujo a una magnitud 20), el que se obtiene del header de al archivo fits (recordar que también necesitamos el valor EXPTIME del header); Este valor se obtiene en forma análoga a lo mostrado anteriormente.\\
Una vez creada la rutina "addStar(hdu,m,RA,DEC)" se pide construir una rutina que lea un catalogo de estrellas y las agregue a una imagen fits, "addStellarCatalog(hdu,catalog)". Para esto se nos ha entregado el catalogo de estrellas "stellar.dat" que consiste en un archivo de texto con el siguiente formato:\\
$\\
"object"\hspace{5pt}RA\hspace{5pt}	DEC\hspace{5pt}	r\_magnitude\hspace{5pt}	SED	index\hspace{5pt}	"star"\\
$\\	
donde los distintos parámetros corresponden a;

\begin{flushleft}
	\label{table:2}      % is used to refer this table in the text
	\begin{tabular}{ll}        % centered columns (4 columns)
	RA&: Ascensión recta en grados. \\
	%\hline   
	DEC&: Declinación en grados.  \\
	%\hline    
	r\_magnitude&: Magnitud del objeto en el filtro R.\\
	%\hline
	SED&: Spectral energy distribution, no es \\ 
	 & importante para esta tarea. \\
	%\hline
	index&: 0, No es importante para esta tarea\\	
\end{tabular}\end{flushleft}
Luego todo lo que necesitamos hacer es crear una rutina que lea la tabla y llame a la rutina "addStar(hdu,m,RA,DEC)", entregándole los parámetros correspondientes.
\subsection{Simulación de Galaxias}
Se nos pide ahora crear una rutina que a\~nada una galaxia a un archivo fit "addGalaxy(hdu,m,RA,DEC,n,Re,el,theta)" que recibe la magnitud, coordenadas, índice de S\'ersic,radio efectivo y elipticidad de una galaxia; la cual modela la galaxia mediante un perfil de S\'ersic, descrito por la siguiente ecuación:
\begin{eqnarray}
	I(\xi) &=& I_0e^{ {-b_n \left( \frac{\xi}{Re}\right)^{1/n}}} \label{sersic}\\
	b_n &=& 2n-0.324
\end{eqnarray}
Donde $I_0$ es la intensidad al centro de la galaxia, Re es  el radio efectivo y representa el radio dentro del cual se encuentra la mitad de la luminosidad total de la galaxia, n es el índice de S\'ersic y describe la forma de decaimiento de la luz de la galaxia, y, $\xi$ es el radio elíptico que toma en consideración la elipticidad $\epsilon$ de la galaxia.
Pero se tiene además que la ecuación (\ref{sersic}) utiliza coordenadas elípticas, las que están descritas por:
\begin{eqnarray}\nonumber
	x &=& \xi cos(E)\\
	y &=& \xi(1-\epsilon)sin(E)
\end{eqnarray}
Donde $\epsilon=1-b/a$ es la elipticidad, $\xi$ el radio elíptico y E la eccentricidad media. Además se nos dice que para el caso general de un objeto centrado en $(x_c,y_c)$ y rotado en un ángulo $\theta$, se tiene:
\begin{eqnarray}\nonumber
	\xi &=&((x-x_c)cos(\theta)+(y-y_0)sin(\theta))^{2}\\
	& & +\frac{((x-x_c)sin(\theta)-(y-y_c)cos(\theta))^2}{(1-\epsilon)^2}\label{e} 
\end{eqnarray} 
Notar que el perfil de S\'ersic depende de $\xi$, Re, n y $I_0$, mientras que la rutina recibe m, RA, DEC, n, Re, $\epsilon$ y $\theta$. $\xi$ se obtiene a partir de la ecuación (\ref{e}) donde $(X_c,y_c)$ es generado a partir de las ordenadas celestes (RA,DEC) transformándolas a coordenadas en la imagen (row,col) mediante la misma transformación utilizada al simular las estrellas (ver sección \ref{estrella}).\\
Nos falta $i_0$, para obtenerlo debemos obtener el numero de cuentas asociadas a la galaxia, utilizando lo descrito en la sección \ref{cuentas} junto con la ecuación de la luminosidad integrada de un perfil de S\'ersic:
\begin{equation}
	l_n=i_0R_e^2\frac{2\pi n}{b_n^{2n}}\Gamma(2n)
\end{equation}
Con esto ya tenemos todo lo que necesitamos para simular una galaxia, sin embargo, notemos que el perfil de S\'ersic va de $\xi=0$ hasta $\xi=\infty$. Ya que no es eficiente y tampoco nos entrega mayor información simular la galaxia en toda la imagen, lo que haremos sera simularla hasta un $\xi'$ suficientemente grande, tal que el valor de $I(\xi')$ sea comparable con el del fondo de la imagen; Por simplicidad a la hora de escribir el código, en lugar de hacer esto en un área circular de radio $\xi'$, lo que haremos sera tomar un cuadrado de lado a centrado en el centro de la galaxia, donde se tomo $a=5R_e$, esta elección se hizo empíricamente observando el resultado de simular varias galaxias para distintos valores de a.
Ya que tenemos listo como simular una galaxia, ahora tenemos que crear una rutina que que lea un catalogo de galaxias y las agregue a una imagen fits ("addGalaxy(hdu,catalog)"), para esto se utilizara el catalogo "galaxy.dat", cuyo formato es:\\
$\\
"object"\hspace{5pt}	RA\hspace{5pt}	DEC\hspace{5pt}	r\_magnitude\hspace{5pt}	SED\hspace{5pt}	redshift\\	"galaxy"\hspace{5pt} n\hspace{5pt}	Re\hspace{5pt}	el\hspace{5pt}	theta\\
$\\
donde los distintos parámetros corresponden a;
\begin{flushleft}
	\label{table:2}      % is used to refer this table in the text
	\begin{tabular}{ll}        % centered columns (4 columns)
	RA&: Ascensión recta en grados. \\
	%\hline   
	DEC&: Declinación en grados.  \\
	%\hline    
	r\_magnitude&: Magnitud del objeto en el filtro R.\\
	%\hline
	SED&: Spectral energy distribution, no es \\ 
	 & importante para esta tarea. \\
	%\hline
	redshift&: corrimiento al rojo, no es \\& importante para esta tarea\\	
	n&: índice de S\'ersic n\\
	Re&: Radio efectivo $R_e$ en píxeles\\
	el&: Elipticidad $\epsilon$\\
	theta&: Ángulo de rotación de la galaxia $\theta$ \\&en radianes
\end{tabular}\end{flushleft}
Todo lo que se necesita hacer es escribir una rutina que lea y reconozca los parámetros del catalogo, y llame a la rutina "addGalaxy(hdu,m,RA,DEC,n,Re,el,theta)" entregándole los parámetros correspondientes
\subsection{Background, PSF y Ruido}
\subsubsection{Background}
Ya que tenemos listas las rutinas para agregar estrellas y galaxias a nuestra imagen fits, ahora se nos pide crear una rutina que le a\~nada un ruido de fondo ("addBackground(hdu,background)"), es decir, que le sume un nivel de intensidad constante a la imagen. Para esto simplemente escribimos una rutina que le sume a todos los píxeles de la imagen el valor "background"
\subsubsection{PSF}
Ahora necesitamos una rutina que convolucione un PSF (point spread function) Gaussiano de desviación estándar $\sigma_{psf}$ con nuestra imagen fits, ("convolvePSF(hdu,$sigma\_PSF$)"), para hacer esto primero nos definimos nuestra PSF como una Gausiana en 2D con $\sigma_x=\sigma_y=\sigma_{PSF}$ y luego la convolucionamos con los valores de nuestra imagen, y finalmente reemplazamos los valores de esta por el resultado de la convolución.
\subsubsection{Ruido}
El ultimo paso para simular nuestra imagen fits, es agregarle ruido, con este objetivo se nos pidió crear la rutina "addNoise(hdu,$sigma\_noise$)" que a\`nade ruido Gaussiano a una imagen con desviación estándar $\sigma_{noise}$, para esto simplemente le agregamos al valor de cada píxel de la imagen un valor aleatorio de distribución Gaussiana con $\sigma=sigma_{noise}$
\subsection{Filtrado}
Posterior a la simulación de nuestra imagen fits, fue necesario aplicar un filtro en el espacio de fourier. Básicamente, a la imagen en el plano de fourier, se le quitaron frecuencias en base a una frecuencia de corte $f\_cut$ de forma uniforme, resultando así una imagen con ruido reducido y mayor contraste de elementos en la misma.

\section{Resultados}

Los resultados de la aplicación de los métodos descritos se puede apreciar en las figuras siguientes. Al momento de modificar variables y parámetros experimentales mendiante la elaboración de este informe, el filtro de la imagen se vio algo estropeado como se muestra en la Figura 6.

\begin{figure}
  \centering
    \includegraphics[width=280pt]{stars.png}
  \caption{Imagen con estrellas}
  \label{fig:ejemplo}
\end{figure}

\begin{figure}
  \centering
    \includegraphics[width=280pt]{galaxys.png}
  \caption{Imagen con galaxias}
  \label{fig:ejemplo}
\end{figure}

\begin{figure}
  \centering
    \includegraphics[width=280pt]{background.png}
  \caption{Imagen con background}
  \label{fig:ejemplo}
\end{figure}

\begin{figure}
  \centering
    \includegraphics[width=280pt]{convolve.png}
  \caption{Imagen convolución}
  \label{fig:ejemplo}
\end{figure}

\begin{figure}
  \centering
    \includegraphics[width=280pt]{noise.png}
  \caption{Imagen con ruido}
  \label{fig:ejemplo}
\end{figure}

\begin{figure}
  \centering
    \includegraphics[width=280pt]{filtered.png}
  \caption{Imagen con filtro}
  \label{fig:ejemplo6}
\end{figure}

\section{Conclusiones}
El objetivo de las tareas planteadas en el informe fue completado en su mayoría. Si bien, como parte del curso en sí, interesa la idea de aprender a manipular datos astronómicos, era también un objetivo lograr familiarizarse con el lenguaje de programación Python y de sus librerías para facilitar la tarea antes mencionada, lo cual se consiguió de forma aceptable. Se logró el cálculo de cuentas, agregar la totalidad de estrellas y galaxias del catálogo, aplicar un $background$ a la imagen, covolucionar una $PSF$, agregar ruido aleatorio, y finalmente, aplicar un filtro para recuperar calidad en la imagen. Lo que quedó pendiente en el desarrollo, fue lo de generar una imagen en el dominio de $Fourier$, dado a complicaciones ténicas que no se logró resolver a tiempo. Sin embargo, logramos comprobar la teoría del procesamiento de imágenes de forma completa, lo que puede apreciarse en las imágenes que planteamos como resultados a nuestra experimentación.


\end{document}